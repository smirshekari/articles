%FIRSR GITHUB
%%%%%%%%%%%%%%%%%%%%%%%%%%%%%%%%%%%%%%%%%
% Large Colored Title Article
% LaTeX Template
% Version 1.1 (25/11/12)
%
% This template has been downloaded from:
% http://www.LaTeXTemplates.com
%
% Original author:
% Frits Wenneker (http://www.howtotex.com)
%
% License:
% CC BY-NC-SA 3.0 (http://creativecommons.org/licenses/by-nc-sa/3.0/)
%
%%%%%%%%%%%%%%%%%%%%%%%%%%%%%%%%%%%%%%%%%

%----------------------------------------------------------------------------------------
%	PACKAGES AND OTHER DOCUMENT CONFIGURATIONS
%----------------------------------------------------------------------------------------

\documentclass[DIV=calc, paper=a4, fontsize=11pt, twocolumn]{scrartcl}	 % A4 paper and 11pt font size

\usepackage{lipsum} % Used for inserting dummy 'Lorem ipsum' text into the template
\usepackage[english]{babel} % English language/hyphenation
\usepackage[protrusion=true,expansion=true]{microtype} % Better typography
\usepackage{amsmath,amsfonts,amsthm} % Math packages
\usepackage[svgnames]{xcolor} % Enabling colors by their 'svgnames'
\usepackage[hang, small,labelfont=bf,up,textfont=it,up]{caption} % Custom captions under/above floats in tables or figures
\usepackage{booktabs} % Horizontal rules in tables
\usepackage{fix-cm}	 % Custom font sizes - used for the initial letter in the document
\usepackage{datetime}

\usepackage{sectsty} % Enables custom section titles
\allsectionsfont{\usefont{OT1}{phv}{b}{n}} % Change the font of all section commands

\usepackage{fancyhdr} % Needed to define custom headers/footers
\pagestyle{fancy} % Enables the custom headers/footers
\usepackage{lastpage} % Used to determine the number of pages in the document (for "Page X of Total")

% Headers - all currently empty
\lhead{}
\chead{}
\rhead{}

% Footers
\lfoot{}
\cfoot{}
\rfoot{\footnotesize Page \thepage\ of \pageref{LastPage}} % "Page 1 of 2"

\renewcommand{\headrulewidth}{0.0pt} % No header rule
\renewcommand{\footrulewidth}{0.4pt} % Thin footer rule

\usepackage{lettrine} % Package to accentuate the first letter of the text
\newcommand{\initial}[1]{ % Defines the command and style for the first letter
\lettrine[lines=3,lhang=0.3,nindent=0em]{
\color{DarkGoldenrod}
{\textsf{#1}}}{}}

%----------------------------------------------------------------------------------------
%	TITLE SECTION
%----------------------------------------------------------------------------------------

\usepackage{titling} % Allows custom title configuration

\newcommand{\HorRule}{\color{DarkGoldenrod} \rule{\linewidth}{1pt}} % Defines the gold horizontal rule around the title

\pretitle{\vspace{-30pt} \begin{flushleft} \HorRule \fontsize{40}{40} \usefont{OT1}{phv}{b}{n} \color{DarkRed} \selectfont} % Horizontal rule before the title

\title{``Where are you from?''\\ A wrong question to ask} % Your article title

\posttitle{\par\end{flushleft}\vskip 0.5em} % Whitespace under the title

\preauthor{\begin{flushleft}\large \lineskip 0.5em \usefont{OT1}{phv}{b}{sl} \color{DarkRed}} % Author font configuration

\author{Saeed Mirshekari \thanks{smirshekari@ift.unesp.br}\; } % Your name
\postauthor{\footnotesize \usefont{OT1}{phv}{m}{sl} \color{Black} % Configuration for the institution name
%Ph.D. in Physics, %Sao P\~{a}ulo, Brazil % Your institution
%smirshekari@wustl.edu %  Your email address
 20$^{\text{th}}$ November, 2013

\par\end{flushleft}\HorRule} % Horizontal rule after the title
\date{} % Add a date here if you would like one to appear underneath the title block


%----------------------------------------------------------------------------------------

\begin{document}

\maketitle % Print the title

\thispagestyle{fancy} % Enabling the custom headers/footers for the first page 

%----------------------------------------------------------------------------------------
%	ABSTRACT
%----------------------------------------------------------------------------------------

% The first character should be within \initial{}
\initial{T}\textbf{his note describes why the author thinks that the popular question of ``where are you from?'' is a wrong question to ask. An alternative would be ``where did you grow up?''.}

%----------------------------------------------------------------------------------------
%	ARTICLE CONTENTS
%----------------------------------------------------------------------------------------

\section*{Why people ask this question?}

The question ``where are you from?'' is a popular question. When people meet each other for the first time it is very probable that they either ask this question directly at the very early stage of the conversation or eventually approach to it indirectly. This question is always in the back of the minds and almost in all situations it is one of the first facts that people would like to know about a new person.

But, why do we ask this question? Sometimes we just use it as a conversation starter but most of the time we ask this question to extract some information. This might be because when we meet each other, naturally, we are curious about the other side's \emph{past} to find the similarities and differences between the two sides.

Smart people like to be efficient. People usually want to extract the most possible information about a stranger via the least possible interaction.  At the first glance, it seems that it is an efficient way to get to know a stranger by asking ``where are you from?''. This might do this job but in a brutal way. If you are not aware of what kind of information you should expect from the answer of this question, lots of valuable information will be lost and many misleading pieces of information might enter into your mind about the person that you think you are trying to know.

%------------------------------------------------

\section*{Why is it a wrong question?}

``Where are you from?'' is an unwise question and sometimes has the potential to be considered as an offensive question as well. First of all, by asking this question, specially at the very early stage of a conversation, you implicitly show how important this issue is for you. The other person could also think that you are trying to categorize people in your mind by the places in which they have been born. This would not be the best first impression that you can give to the person that you just met. Secondly, what do you really mean by ``where are you \emph{from}?''? You certainly doesn't mean that to where does she/he belong. If it is so, this is definitely offensive. What you really mean probably is ``where did you grow up?''.

In addition, if you really want to know the person, it could also be a poor and an unwise question for this purpose. Asking ``where are you from?'' gives you only a little information about the person that you want to know. In the most accurate case, the place in which one was born in, grown up and has lived for a while might indicate some general facts about \emph{the majority} of the people who live there. But these facts are not what you want to learn about that particular person that you are talking to at the moment, unless you are meeting a senator or a president. In this case you may expect some correlations between her/him and the majority of the people who have voted for her/him. Otherwise, asking ``where are you from?'' would be an unwise choice to discover different personal aspects of an individual.

Besides, this question might also give you unreliable information. Suppose you meet somebody in a party. After greetings, you start a conversation and ask ``where are you from?''. ``I'm from Mississippi'', he replies. Even though you know that Mississippi has the highest rate of obesity of any U.S. state \cite{USobesity}, you never think he is obese. Because you can see him! He is skinny. On the other hand, you can basically know nothing about his religious beliefs and habits neither by staring at him nor knowing that he is from Mississippi. Although you know that 63\% of Mississippians attend church weekly or almost weekly \cite{church}, this piece of information about the majority of Mississippians doesn't say \emph{anything} about the religious beliefs of this person. To find out if this person goes to church often or not you must change the question.

If you really care about knowing the person you meet, you will certainly need much more information rather than the place of birth. It will be including his major interests, his expertise, the experiences he has had, the books he has read, the places he has seen, the people he has met and interacted with, etc. Among many others, these are the facts that give you much more valuable and reliable information about the person rather than the place of birth. Any answer to the question ``where are you from?'' gives no accurate information about the person except a few ones, like the language that she/he might speak and the historical local events that he may have experienced when she/he was living there. 


%------------------------------------------------

\section*{Is there any better alternative?}
Yes, a better option is ``where did you grow up?''. By asking this question, you point right to the target. This alternative gives you all the information that you could get out of ``where are you from?'' but in a much clearer and nicer way with no possible offense. Also, it clearly states that you are just curious about the place that she/he grew up in and also send a signal that you are aware of the fact that whatever the answer will be, this will be just one single piece of information; there are many more that you can talk about them afterward. 

%----------------------------------------------------------------------------------------
%	REFERENCE LIST
%----------------------------------------------------------------------------------------

\begin{thebibliography}{99} % Bibliography - this is intentionally simple in this template

\bibitem{USreligion} ``Religious Composition of the U.S.''. U.S. Religious Landscape Survey. Pew Forum on Religion \& Public Life. 2007. Retrieved October 23, 2008.
\bibitem{USobesity} ``U.S. Obesity trends''. Center for Disease Control. Retrieved 23 July 2011.

\bibitem{church} ``http://www.gallup.com''. Frank Newport, 17 February 2010.
 
\end{thebibliography}

%----------------------------------------------------------------------------------------

\end{document}
